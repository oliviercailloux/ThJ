\RequirePackage[l2tabu, orthodox]{nag}
\documentclass[french]{beamer}
\input{preamble/packages}
\input{preamble/redac}

%\setbeamertemplate{headline}[singleline]
%\setbeamertemplate{footline}[authortitle]

\title{Les théories de la justice}
\subject{Philosophie politique}
\keywords{Rawls, égalitarisme, utilitarisme}
\author{Léa Meissonnier \and Olivier Cailloux}
\institute[Paris-Dauphine]{Université Paris-Dauphine}
\date{\formatdate{24}{1}{2025}}
  
\begin{document}
\begin{frame}[plain]
	\tikz[remember picture,overlay]{
		\path (current page.south west) node[anchor=south west, inner sep=0] {
			\includegraphics[height=8mm]{Dauphine-Noir.png}
		};
		\path (current page.south east) node[anchor=south east, inner sep=0] {
			\includegraphics[height=1cm]{LAMSADE95.jpg}
		};
		\path (current page.south) ++ (0, 4em) node[anchor=south, inner sep=0] {
			\scriptsize\textcolor{blue}{\url{https://github.com/oliviercailloux/ThJ/}}
		};
	}
	\titlepage
\end{frame}
\addtocounter{framenumber}{-1}

\begin{frame}
	\frametitle{\translate{Outline}}
	\tableofcontents[hideallsubsections, sectionstyle=shaded/show]
\end{frame}

\AtBeginSection{
	\begin{frame}
		\frametitle{\translate{Outline}}
		\tableofcontents[currentsection]
	\end{frame}
}

\section{Introduction}
% Introduction (2/3 min) - vise à simplement introduire le sujet dans ses très grandes lignes (en définissant très simplement la notion de justice, définition de Montesquieu ?) et présenter le plan
% Partie 1 (7 min) - les théories de la justice
% Principes : égalité, … Test : cohere with, and help illuminate, our considered convictions of justice (Goodman, Rawls) : encourage l’esclavagisme? (Williamson)
%   Philosophie politique normative : just or good. Les institutions. Philosophie politique : analyse conceptuelle de la puissance, de la souveraineté, la nature de la loi.
%   Philosophie morale. Un continuum ? Ligne de division : sphère privée (justice utilitariste n’en dit pas assez), sphère publique (éthique du care ne la remplace pas)
% Partie 2 (7/9 min) - L’utilitarisme
% Betham, Mill…
% Partie 3 (7 min) - L’égalitarisme libéral
% Rawls, Dworkin, Sen, Anderson…
% Partie 4 (5/7 min) – Théories contemporaines 
% Féminisme (Nussbaum), multiculturalisme (Kymlicka)
% Conclusion (2 min)

\begin{frame}
	\frametitle{Introduction}
	\begin{itemize}
		\item To do
	\end{itemize}
\end{frame}

\section{Les théories de la justice}
\subsection{Termes}
\begin{frame}
	\frametitle{La justice}
  Justice : plusieurs sens dans le langage commun
	\begin{itemize}
		\item Judiciaire, conforme à la loi : ce malfrait a échappé à la justice ; la justice dans le pays X est organisée comme suit…
		\item Équité, ce qui est juste : il n’est pas juste que…
		\item Distincts : loi injuste
	\end{itemize}
	\begin{block}{La loi et la justice (attribuée à Montesquieu)}
    \emph{Une chose n’est pas juste parce qu’elle est loi ; mais elle doit être loi parce qu’elle est juste.}
  \end{block}
  Ici : justice au sens d’équité {\small (dans le sens général du terme, pas nécessairement celui de Rawls)}
\end{frame}

\begin{frame}
	\frametitle{Les théories}
	\begin{itemize}
		\item Théorie de la justice : tentative de description de ce qui est juste
    \item Exemple : ressources selon la taille
    \item Mise en ordre de nos intuitions sur ce qui est juste
		\item Théorie physique décrit le monde physique
		\item Ici, le sujet décrit n’existe pas dans un sens aussi fort
    \item Cependant, il pré-existe dans un sens révisable
    \item La théorie implique des institutions : distributions des ressources, organisation de la loi, …
    \item Exemples : variantes du communisme, du capitalisme, …
	\end{itemize}
\end{frame}

\subsection{Méthode}
\begin{frame}
	\frametitle{Des principes}
	\begin{itemize}
		\item Égalité (socialisme ?)
		\item Liberté (libertarianisme ?)
		\item Valeurs fondamentales distinctes donc théories irréconciliables ?
		\item Solution \citep{dworkin_taking_1978} : égalité au sens large ?
	\end{itemize}
\end{frame}

\begin{frame}
	\frametitle{Tests}
	\begin{itemize}
		\item Falsification ?
		\item Cohérence avec nos convictions de justice \citep{goodman_fact_1983, rawls_theory_1999}
		\item Intuitions a priori
		\item Conclusions guidées par la théorie
		\item Théorie peut modifier nos intuitions
		\item Intuitions peuvent rejeter la théorie
	\end{itemize}
\end{frame}

\subsection{Relations}
\begin{frame}
	\frametitle{Relations}
	\begin{itemize}
		\item Philosophie politique normative : étudie (entre autres) les institutions qu’il est bon de mettre en place
		\item Philosophie morale : la façon dont il est bon de se comporter (individuellement ou pas)
		\item Un continuum \citep{kymlicka_contemporary_2001} ? %Ligne de division : sphère privée (justice utilitariste n’en dit pas assez), sphère publique (éthique du care ne la remplace pas)
	\end{itemize}
\end{frame}

\section{L’utilitarisme}
\begin{frame}
	\frametitle{Title}
	\begin{itemize}
		\item Item
	\end{itemize}
\end{frame}

\section{L’égalitarisme libéral}
\begin{frame}
	\frametitle{Title}
	\begin{itemize}
		\item Item
	\end{itemize}
\end{frame}

\section{Les théories contemporaines}
\begin{frame}
	\frametitle{Title}
	\begin{itemize}
		\item Item
	\end{itemize}
\end{frame}

\section{Conclusion}
\begin{frame}
	\frametitle{Title}
	\begin{itemize}
		\item Item
	\end{itemize}
\end{frame}

\begin{frame}[plain]
	\addtocounter{framenumber}{-1}
	\begin{center}
		\huge
		\textit{Merci pour votre attention !}
	\end{center}
\end{frame}

\appendix
\AtBeginSection{
}

\begin{frame}[allowframebreaks]
	\frametitle{\refname}
  \bibliography{philoeco}
\end{frame}

\clearpage\pdfbookmark{License}{License}
\begin{frame}[plain]
	\frametitle{License}
	This presentation, and the associated \LaTeX{} code, are published under the \href{https://opensource.org/licenses/MIT}{MIT license}. Feel free to reuse (parts of) the presentation, under condition that you cite the author.
	
	Credits are to be given to \hrefblue{https://www.lamsade.dauphine.fr/~ocailloux/}{Olivier Cailloux}, Université Paris-Dauphine.
\end{frame}
\addtocounter{framenumber}{-1}
\end{document}

\begin{frame}
	\frametitle{Title}
	\begin{itemize}
		\item Item
	\end{itemize}
\end{frame}

\begin{frame}
	\frametitle{Title}
	\begin{block}{Block}
%		\setlength\abovedisplayskip{1 ex}% reduce space above equations
		\begin{itemize}
			\item Item
		\end{itemize}
	\end{block}
	\begin{itemize}
		\item Item
	\end{itemize}
\end{frame}

